%!TEX program = xelatex
\documentclass[cn,hazy,blue,14pt,screen]{elegantnote}
\title{2020级物理第一次班会}
\author{冷轩}

%\version{2.30}
\date{\zhtoday}



\begin{document}
\maketitle

\centerline{
  \includegraphics[width=0.4\textwidth]{nbu1.png}
}

\newpage
\tableofcontents   \label{content}
\phantomsection%防止目录中页码和pdf链接中的页码不一致,而超链失效。



\newpage
\section{班委选举}

{\centering \Large 班委选举}


\newpage
\section{班主任的定位}
自己对大学班主任角色的定位,就两点:
\begin{itemize}
\item 站在学生的角度上,和班上学生一条阵线;
\begin{itemize}
\item 不整虚的,讲实在话;
\item ... ...
\end{itemize}

\item 辅助班上学生顺利毕业
\begin{itemize}
\item 辅助走得高的,走的更高远;辅助走得慢的,跟上;
\item 帮助引进提升的机会,如比赛和科研;
\item ... ...
\end{itemize}

\item 教课和班主任基本是用爱发电,珍惜关心你们的老师~~
\end{itemize}



\newpage
\section{大学生的几点重要的思维转变}
\subsection{成年人意识}
{\bf 大学生基本是准成年人和成年人了。}

\begin{itemize}
\item  成年人的第一个要点:为自己的行为负责,为自己的前途命运负责。

\item  进而引出一个重要的意识:主动。

\item  主动:让自己忙碌起来,时间用在刀刃上、主动抓住机会。
\end{itemize}


{\bf 外部环境发生变化:父母不在身边、老师不再贴身督促。}
\begin{itemize}
\item  进而引出另一个重要的意识:自觉、自律
\end{itemize}

 
 {\bf 做人做事的态度(很多时候机会只有一次)}
 \begin{itemize}
\item 思考未来,思考自己的位置;
 
\item 第一印象很关键;

\item 守时是基本;

\item 打开格局。
\end{itemize}
 
 {\bf 一开始就要有毕业愿景}(要有大方向,不清晰没关系,及时明确)
 
 
 
 
 \newpage
 \subsubsection{问“问题”与听取建议、意见}
\paragraph{听取建议、意见}
夫子说过“三人行必有我师焉”。无论是前辈、同学、同事、晚辈沟通,都有听取意见的时候。听取他人意见不是听取他人对自己的观点的附和,往往是与自己观点不同甚至相背的。这时候就让人很难受。但是难受是让自己进步的前兆。如果非常坚持自己的观点,那就给他人观点一个机会,先试用一下他人观点,比较一下优缺点。最终结果往往不是自己或他人观点的胜出,而是不同观点的融合。所以要经常反思:我给过别人建议、意见机会了吗?

\paragraph{RTFM} is an initialism for the expression "read the fucking manual" – typically used to reply to a question that could have been easily answered from the product manual or documentation.\footnote{\url{https://en.wikipedia.org/wiki/RTFM}}

大家的时间和精力都是有限的,请不要问些“简单问题”浪费别人的生命。每个人都应该具备自主学习能力,不要张口就要“答案”,这是不利于自己成长的。问别人“问题”前请反问自己两点:
\begin{itemize}
\item 网上搜索过了吗?
\item 自己思考过了吗?
\item 流程:首先尝试自己解决,再和同事讨论,再问上级。
\item 记录解决问题过程的心得体会,然后自己解决问题的能力与速度会不断提升,进而关注到别人未解决的问题,此时你已经站到人类知识的边界了,为人类拓荒了。
\end{itemize}

 
 

\newpage
\subsection{课程学习}
{ \bf 四字核心:详略得当}
\begin{itemize}
\item 发达国家名校的课程学习:一般课上松,考试难

\item 除申请国外研究生需要刷分数绩点外,毕业就业和考研一般只认学位,几乎没有关心你本科课程分数的

\item 人的精力是有限的,以自己未来发展方向为中心,合理调整各个课程花费的时间,水课划水,自己核心课认真学习

\item 全面扎实的基本概念最重要,并不需要特别深入,也不会像高考一样变着花子考你。在你入职或升学后,自然而然会要你深入

\item 注重积累与沉淀,形成自己的东西(记笔记和积累程序是个好方法,记笔记推荐用\LaTeX )
\end{itemize}


{\bf 英语要过六级,兼顾学习生活英语(参考我写的文档)}


{\bf 作为物理班,要会一门编程语言,推荐Python,无论是工作中还是科研中都非常强大}


\newpage
\paragraph{素描学习法}
这种方法是从网络视频,普通物理-台湾国立交通大学-李威仪,那学来的。其理论依据应该可以用马克思关于真理和认识事物的相应理论。认识事物是不断反复的过程。素描的过程就是先画事物的轮廓,然后再不断丰富细节。

\paragraph{正确做题}
做题是很重要的。自然界是没有答案给你的。平时做一道题无疑是一次小小的探索。把自己有的想法完全写出来,这就是思维的过程,这个很重要。平时做题就是解决问题,解决问题可以动用身边的一切资源。做题就是,大胆的查找资料(不是找直接的答案)。平时做题和考试做题不同,平时做题是知识的运用和方法的训练过程,考试做题是知识的运用和方法掌握的检测过程,本质是不一样的。



\newpage
\section{基本技能}
{\bf 编程语言学习:一本书在手(了解基本结构)+网上搜索资料}

\subsection{Zotero文献管理安装与简介}
现场演示!

\newpage
\subsection{\LaTeX 安装与简介}
现场演示! Texlive 与 Texworks 或 overleaf!

\newpage
\subsection{Python 安装与简介}
Anaconda套装!现场演示!


\newpage
\section{科研与考研}
{\bf 明确定位很重要!}
\begin{itemize}
\item 拿学位找工作和继续深造拿博士学位,完全不一样!!!

\item 是否要继续博士?很关键,直接影响你自己的方向和导师培养。
\end{itemize}


\newpage
\section{自由讨论}

















\end{document}