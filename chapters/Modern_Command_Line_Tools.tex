\chapter{现代高效的命令行工具}

\section{Shell}
\label{sec:shell}

\subsection{修改默认Shell}

\url{https://www.cyberciti.biz/faq/change-my-default-shell-in-linux-using-chsh/}


\subsection{ZShell}
\label{sec:zshell}

基本入门配置以及使用说明请参照项目档案: Oh-My-Zsh,\url{https://github.com/ohmyzsh/ohmyzsh}

简单说明一下 ZShell 比 Bash 体验更好的方面:
\begin{itemize}
\item 命令或路径自动补全
\item 丰富的插件机制
\item Shell theme,i.e., powerlevel10k(\url{https://github.com/romkatv/powerlevel10k})
\end{itemize}

\subsection{Fish Shell}
配置请参考:\url{https://github.com/oh-my-fish/oh-my-fish}

\section{命令行工具}

\begin{itemize}
\item fd: 文件查询,可以替代 find, \url{https://github.com/sharkdp/fd}
\item fasd: 一键跳转到常用的文件所在目录, \url{https://github.com/clvv/fasd}
\item ripgrep/ripgrep-all: 内容查询, 可以替代 grep, \url{https://github.com/BurntSushi/ripgrep}
\item fzf: 模糊查询,极大改善命令交互体验, \url{https://github.com/junegunn/fzf}
\item tldr: 查询陌生命令用法,部分替代 man, \url{https://github.com/tldr-pages/tldr}
\end{itemize}

\label{sec:tools}


\section{文本编辑}
\subsection{nano}


\subsection{Vim}
安装:\verb|sudo apt install vim|

\subsubsection{常用配置}
\begin{verbatim}
set nu
set fileencodings=utf-8,gbk,default,latin1
set encoding=utf-8
let $LANG="zh_CN.UTF-8"
\end{verbatim}


\subsubsection{一般模式(命令模式)}
\paragraph{打开文件、保存、关闭文件}
\begin{itemize}
\item vi filename :打开filename文件
\item w :保存文件
\item w vpser.net :保存至vpser.net文件
\item q :退出编辑器,如果文件已修改则用下面的命令
\item q! :退出编辑器,且不保存
\item wq :退出编辑器,且保存文件
\end{itemize}


\paragraph{常用光标移动命令}
\begin{itemize}
\item {\color{red}注意区分大小写}
\item h:左;j:下;k:上;l:右;这很符合书写习惯,用下就知道了
\item n<hjkl>移动相应字符
\item 0:零,移动到这一行最前面的字符处
\item \$ :移动到这一行最后面的字符处
\item G:移动到文件的最后一行
\item nG:n表示数字,移动到这个文件的第n行
\item n<Enter>:n表示数字,光标向下移动n行,等效于nj
\item 空格键:向右
\item Backspace :向左
\item Enter :移动到下一行首
\item - :移动到上一行首
\end{itemize}


\paragraph{搜索与替换}
\begin{itemize}
\item /word:从光标位置开始,向下搜索一个名为“word”的字符
\item ?word:从光标位置开始,向上搜索一个名为“word”的字符
\item n:表示“next”
\item N:表示“previous”
\item \%s/word1/word2/g:表示将word1替换为word2,g表示全局替换
\item \%s/word1/word2/gc:其中多的一个c表示替换之前有个用户提示,需要确认
\end{itemize}


\paragraph{删除、复制与粘贴}
\begin{itemize}
\item x,X:小写x表示向后删除字符;大写X表示向前删除字符
\item nx:表示向后删除n个字符
\item dd:表示删除光标所在的整行
\item ndd:表示包括光标所在行向下删除n行
\item d1G:表示删除当前行到第1行所有的数据
\item dG:删除光标所在位置到最后一行所有的数据
\item yy:复制光标所在的一整行,也可以用 ayy 复制,a 为缓冲区,a也可以替换为a到z的任意字母,可以完成多个复制任务。
\item nyy:n表示数字,复制光标所在的向下n行,也可以用 anyy 复制,a 为缓冲区,a也可以替换为a到z的任意字母,可以完成多个复制任务。
\item p,P:p将已复制的数据粘贴到光标下一行,P相反,如果使用了前面的自定义缓冲区,建议使用“ap”或“aP”进行粘贴。
\item J:将光标所在的行与下一行结合在一起
\item u,U:小写u撤销上一步操作,大写U撤销对当前行的所有操作
\item Ctrl+r:重做上一个操作,恢复
\item .:重复前一个动作
\item :复制单个字符:首先进入正常模式(一般模式)然后按v,进入visual方式,然后就可以移动方向键选中文本,然后按y,就拷贝完成。
\end{itemize}


\subsubsection{编辑模式}
\begin{itemize}
\item i,I:i从当前光标所在处插入,I从当前所在行的第一个非空格处开始插入
\item a,A:a从当前光标所在处下一个字符开始插入,a从当前所在行的最后一个非空格处开始插入。A是从所在行尾插入。
\item o,O:o在当前光标所在的下一行插入一个行行,O相反
\item r,R:r会替换光标所在的那一个字符,R会一直替换光标所在的字符
\item J :合并光标所在行及下一行为一行(命令模式)
\end{itemize}



\subsubsection{vim中使用系统粘贴板}
%\url{http://www.linuxsir.org/bbs/thread344622.html}
用vim这么久了,始终也不知道怎么在vim中使用系统粘贴板,通常要在网上复制一段代码都是先gedit打开文件,中键粘贴后关闭,然后再用 vim打开编辑,真的不爽;上次论坛上有人问到了怎么在vim中使用系统粘贴板,印象里回复很多,有好几页的回复却没有解决问题,今天实在受不了了又在网上找办法,竟意外地找到了,贴出来分享一下。 

如果只是想使用系统粘贴板的话直接在输入模式按Shift+Inset就可以了,下面讲一下vim的粘贴板的基础知识,有兴趣的可以看看,应该会有所收获的。 
vim帮助文档里与粘贴板有关的内容如下: 

1. vim有12个粘贴板,分别是0、1、2、...、9、a、“、+;用:reg命令可以查看各个粘贴板里的内容。在vim中简单用y只是复制到“(双引号)粘贴板里,同样用p粘贴的也是这个粘贴板里的内容; 

2. 要将vim的内容复制到某个粘贴板,需要退出编辑模式,进入正常模式后,选择要复制的内容,然后按"Ny完成复制,其中N为粘贴板号(注意是按一下双引号然后按粘贴板号最后按y),例如要把内容复制到粘贴板a,选中内容后按"ay就可以了,有两点需要说明一下: 
* “号粘贴板(临时粘贴板)比较特殊,直接按y就复制到这个粘贴板中了,直接按p就粘贴这个粘贴板中的内容; 
* +号粘贴板是系统粘贴板,用"+y将内容复制到该粘贴板后可以使用Ctrl+V将其粘贴到其他文档(如firefox、gedit)中,同理,要把在其他地方用Ctrl+C或右键复制的内容复制到vim中,需要在正常模式下按"+p; 

3. 要将vim某个粘贴板里的内容粘贴进来,需要退出编辑模式,在正常模式按"Np,其中N为粘贴板号,如上所述,可以按"5p将5号粘贴板里的内容粘贴进来,也可以按"+p将系统全局粘贴板里的内容粘贴进来。 
